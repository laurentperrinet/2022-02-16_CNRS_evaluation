% %!TeX TS-program = Lualatex
% %!TeX encoding = UTF-8 Unicode
% %!TeX spellcheck = fr-FR
% %!BIB TS-program = biber
% % -*- coding: UTF-8; -*-
% % vim: set fenc=utf-8
% \documentclass[9pt,french,a4paper,oneside]{article}%,twoside,draft
% %============ common ===================
% \usepackage[utf8]{inputenc}%
% \usepackage[french]{babel}%
% \usepackage{csquotes}%
% \usepackage{etaremune}%
% \usepackage[final]{pdfpages}%
% %% --- unit formatting ---
% \usepackage{microtype}	% Better typography \tolerance 1414
% \hbadness 1414
% \emergencystretch 1.5em
% \hfuzz 0.3pt
% \widowpenalty=10000
% \vfuzz \hfuzz
% \raggedbottom
% %%%%%%%%%%%%%%%%%%%%%%%%%%%%%%%%%%%%%%%%%%%%%%%%%%%%%%%%%%%%%%%%%%%%%
% \usepackage{amsmath,amsfonts,amsthm}			% Math packages
% \usepackage{bm}%This package defines commands to access bold math symbols
% \usepackage{graphicx}					% Enable pdflatex
% %============ front-matter ===================
% \newcommand{\BookTitle}{Centre National de la Recherche Scientifique}
% \newcommand{\Title}{%
% Liste complète des publications
% }%
% \newcommand{\SubTitle}{%
% Pour évaluation par les sections du Comité national}%
% \newcommand{\Author}{Laurent U~Perrinet}%
% \newcommand{\Team}{\'Equipe NEural OPerations in TOpographies (NeOpTo)}%
% \newcommand{\Institute}{Institut de Neurosciences de la Timone}%
% \newcommand{\InstituteUMR}{UMR 7289, CNRS~/~Aix-Marseille Université}%
% \newcommand{\Address}{27, Bd. Jean Moulin, 13385 Marseille Cedex 5, France}
% \newcommand{\Website}{https://laurentperrinet.github.io/}
% \newcommand{\Email}{Laurent.Perrinet@univ-amu.fr}
% %%%%%%%%%%%%%%%%%%
% %BIBLIOGRAPHY
% %%%%%%%%%%%%%%%%%%
% \usepackage[natbib=true,
% 			bibencoding=utf8,
% %			encoding=utf8,
% 			maxcitenames=10,
% 			maxnames=10,
% 			%minnames=3,
% 			maxbibnames=10,
%             sortcites=true,
%             block=space,
%             backend=biber,
% %            doi=false,isbn=false,url=false,
%             citestyle=alphabetic,%authoryear-comp,
%             backref=true,
% %            bibstyle=alphabetic
%             ]{biblatex}%
% \addbibresource{~/metagit/blog/perrinet_curriculum-vitae_tex/LaurentPerrinet_Publications.bib}
% \addbibresource{~/metagit/blog/perrinet_curriculum-vitae_tex/LaurentPerrinet_Presentations.bib}
% % https://tex.stackexchange.com/questions/359311/how-to-underline-name-of-specific-authors-in-biblatex
% \usepackage{xpatch}
% \usepackage[normalem]{ulem}
%
% \newbox\savenamebox
%
% %\newbibmacro*{name:bbold}[2]{%
% %  \def\do##1{\iffieldequalstr{hash}{##1}{\bfseries\setbox\savenamebox\hbox\bgroup \listbreak}{}}%
% %  \dolistloop{\boldnames}%
% %}
% \newbibmacro*{name:bbold}[2]{%
%   \def\do##1{\iffieldequalstr{hash}{##1}{\setbox\savenamebox\hbox\bgroup \listbreak}{}}%
%   \dolistloop{\boldnames}%
% }
%
% \newbibmacro*{name:ebold}[2]{%
%   \def\do##1{\iffieldequalstr{hash}{##1}{\egroup\uline{\usebox\savenamebox}\listbreak}{}}%
%   \dolistloop{\boldnames}%
% }
%
% \xpatchbibmacro{name:given-family}{\usebibmacro{name:delim}{#2#3#1}}{\usebibmacro{name:delim}{#2#3#1}\begingroup\usebibmacro{name:bbold}{#1}{#2}}{}{}
%
% %\xpretobibmacro{name:given-family}{\begingroup\usebibmacro{name:bbold}{#1}{#2}}{}{}
% \xapptobibmacro{name:delim}{\begingroup\normalfont}{}{}
%
% \xapptobibmacro{name:given-family}{\usebibmacro{name:ebold}{#1}{#2}\endgroup}{}{}
% \xapptobibmacro{name:delim}{\endgroup}{}{}
%
% \newcommand*{\boldnames}{}
% \forcsvlist{\listadd\boldnames}{
%   {8fa031352aa34db95f5c1022358042a3}, % InBold
%   {01b588ba4e4ad753feae6c81709fc04b}} % Highlight
%
% %%%%%%%%%%%%%%%%%%%%%%%%%%%%%%
% %% OPTIONAL MACRO FILES
% %============ hyperref ===================
% \usepackage[unicode,linkcolor=blue,citecolor=blue,filecolor=black,urlcolor=blue,%pdfborder={0 0 0}
% ]{hyperref}%
% \hypersetup{%
% unicode = true, %
% pdftitle={\Title},%
% pdfauthor={\Author < \Email > - \Institute, \InstituteUMR , \Address - \Website},%
% pdfsubject={\Title}%
% }%
% %\hypersetup{linkcolor=blue,citecolor=blue,filecolor=black,urlcolor=blue}
% %\hyphenpenalty=5000
% %\tolerance=1000
% %% DOCUMENT LAYOUT
% %\usepackage{geometry}
% %\geometry{a4paper, textwidth=5.5in, textheight=8.5in, marginparsep=7pt, marginparwidth=.6in}
% %\setlength\parindent{0in}
% %% ---- MARGIN YEARS
% \newcommand{\years}[1]{\marginpar{\textit{\scriptsize #1}}}
% \providecommand{\natexlab}[1]{#1}
% %\providecommand{\url}[1]{\texttt{#1}}
% %\expandafter\ifx\csname urlstyle\endcsname\relax
%  \providecommand{\doi}[1]{doi: #1}%\else
% %  \providecommand{\doi}{doi: \begingroup \urlstyle{rm}\Url}\fi
% %\makepagestyle{chapter}
% %\makeoddfoot{chapter}{\addRevisionData}{\thepage}{}
% %\makeevenfoot{chapter}{\addRevisionData}{\thepage}{}
% \usepackage{geometry}
% \geometry{hscale=0.66,vscale=0.8,centering}
% %%%%%%%%%%%% Her begynner selve dokumentet %%%%%%%%%%%%%%%
% \begin{document}
%
% \begin{titlepage}
%
% \begin{center}
% %\vskip 2cm
% \emph{\Large \BookTitle }%
% \vskip 2cm
% {\Huge \Title}
% \vskip 1cm
% {\Large Laurent U~\textsc{Perrinet}}
% \vskip 1cm
% \emph{\Large \SubTitle }\\%
% %\begin{tabular}[t]{ccc}
% %\includegraphics[height=2.5cm]{logo_cnrs-UAM.png} &%\includegraphics[]{}
% %\includegraphics[height=2.5cm]{logo_int.png} %& %
% %%	\includegraphics[height=2.5cm]{U2_logo.pdf}
% %\end{tabular}
% \vskip 1cm
% \includegraphics[width=1.\textwidth]{/Users/laurentperrinet/quantic/libraries/slides.py/figures/troislogos.png}
% \vskip 1cm
% \begin{tabular}[t]{|c|}
% \hline
% \Team \\\hline
% \Institute \\ \InstituteUMR \\\hline
% \Address \\\hline
% \url{\Website}\\\hline
% \url{\Email}\\\hline
% \end{tabular}
% \vskip .5cm
% \vfill
% {\large 16 Février 2022}\\
%
% \end{center}
% \end{titlepage}
%
% %%%%%%%%%%%% Endyet daag koepf dokumentet %%%%%%%%%%%%%%%

%\chapter{Liste complète des publications}
%\label{chap:publis}%

% TODO : La Section recommande d’indiquer, à la suite de chaque publication, l’impact facteur et/ou le quartile de la revue dans la base de données indexée.

%\section{Liste des publications}
%%
%Vous présenterez la liste de vos publications scientifiques depuis les 5 ou 10 derniers semestres (distincte de la liste exhaustive de vos publications, mise à jour et transmise en fichier séparé) selon l?une des deux listes suivantes :

% \section{Articles de revues en cours de révision}%
%Publications dans des journaux à comité de lecture}
\begin{enumerate}
%
% \item[A51] \fullcite{Benvenuti20}%

\item[A50] \fullcite{Ladret22}% _LA_
% Create unordered list in LaTeX
\begin{itemize}
  \item Cette publication en révision (deuxiémé révision) combine 1/ une création de stimuli basée sur un modèle, 2/ des enregistrements Neuro physiologiques basés sur ce protocole novateur et 3/ une méthodologie nouvelle de décodage. Basés sur ces innovations théoriques et expérimentales, nous avons mis en évidence le rôle de la précision dans le traitement dynamique de l'information, et donc le rôle potentiel de processus prédictifs dans des données biologiques.
\end{itemize}
% \item[A49] \fullcite{Franciosini21}% _LA_

\item[A48] \fullcite{Grimaldi22pami}% _LA_
\begin{itemize}
  \item Nous avons combiné dans cette publication une approche d'ingénierie des systèmes neuromorphiques et une approche de modélisation des neurosciences computationnelles. L'objectif est de permettre une catégorisation rapide d'objets à partir de flux d'événements. Nous utilisons les données de caméras événementielles et une innovation a été de caractériser de manière extensive l'efficacité de ce réseau en fonction de différents paramètres d'entrée. La nouveauté était de rapprocher ce type de modèle des réseaux événementiels ou des réseaux de neurones à impulsions.
\end{itemize}

% \end{enumerate}
%
% \section{Articles de revues internationales à comité de lecture}%
%
% \begin{enumerate}

\item[A47] %\years{2022}
\fullcite{Chavane22}%
\begin{itemize}
  \item Dans cette revue de l'état de l'art sur l'anatomie du cortex visuel primaire, nous avons proposé une hypothèse novatrice sur l'organisation de la formation sur la surface de cette aire.
\end{itemize}

\item[A46] %\years{2020}
\fullcite{BoutinFranciosiniChavaneRuffierPerrinet20}% _LA_
\begin{itemize}
  \item Ce travail étend l'architecture habituellement rencontrée en apprentissage profond en incluant des processus prédictifs. Un résultat principal est d'obtenir après apprentissage un réseau dont la structure est explicable, un résultat majeur par rapport au réseau profond classique. Un second résultat est de montrer le parallèle entre ce type de réseau et ce que l'on peut observer dans le cortex visuel primaire.
\end{itemize}
%
% \item[A45] \fullcite{BoutinFranciosiniRuffierPerrinet20feedback}% _LA_

\item[A44] \fullcite{Dauce20} %  _LA_
\begin{itemize}
  \item L'objectif de ce travail est d'introduire des processus actifs dans les modèles actuels de traitement d'images. Inspirés par ce qui peut être observé dans les systèmes biologiques, nous avons introduit des saccades oculaires pour résoudre le problème conjoint de l'identification et de la localisation d'objets dans les images. Le résultat principal est de pouvoir réaliser cette tâche avec beaucoup moins de ressources informatiques.
\end{itemize}

\item[A43] \fullcite{PasturelMontagniniPerrinet20}%  _LA_
\begin{itemize}
  \item Nous avons un ici à introduit un nouveau paradigme dans lequel des protocole expérimentaux pouvait changer à des mouvements inpredictible. ce travail combine une analyse théorique et des résultats au comportementaux qui permettent de mettre à jour nos connaissances sur le traitement de la volatilité dans des données comportemental.
\end{itemize}

% \item[A42] \years{2019}\fullcite{Perrinet19hulk}% _FA_
%
% \item[A41] \fullcite{Ravello19}%
%
% \item[A40] \fullcite{Chemla19}
%
% \item[A39] \years{2018} \fullcite{Damasse18}
%
% \item[A38] \fullcite{Vacher16}

\item[A37] \years{2017} \fullcite{KhoeiMassonPerrinet17}%%  _LA_
%
% \item[A36] \years{2016} \fullcite{Kremkow16}%
%
% \item[A36] \years{2015} \fullcite{Taouali16}% %  _LA_
%
% \item[A35] \fullcite{Vacher15nips}%
%
% \item[A34] \fullcite{PerrinetBednar15}% _FA_

\item[A33] \years{2013} \fullcite{PerrinetAdamsFriston14}% _FA_
%
% \item[A32] \years{2013} \fullcite{Khoei13jpp}%%  _LA_
%
% \item[A31] \fullcite{Kaplan13}%
%
% \item[A30] \fullcite{Nava13}%
%
% \item[A29] \years{2012} \fullcite{Simoncini12}%
%
% \item[A28] \fullcite{Perrinet12pred}% _FA_
%
% \item[A27] \fullcite{Sanz12} % %  _LA_
%
% \item[A26] \fullcite{Friston12}
%
% \item[A25] \fullcite{Adams12} %
%
% \item[A24] \fullcite{Masson12}%  _LA_
%
% \item[A23] \fullcite{Voges12}%  _LA_
%
% \item[A22] \years{2011} \fullcite{Fleuriet11}
%
% \item[A21] \fullcite{Bogadhi11}
%
% \item[A20] \years{2010} \fullcite{Perrinet10shl}% _FA_
%
% \item[A19] \fullcite{Dauce10}%  _LA_
%
% \item[A18] \fullcite{Voges10jpp}%  _LA_
%
% \item[A17] \fullcite{Kremkow10jcns}
%
% \item[A16] %\fullcite{Kremkow10jcns}
% Khaled Masmoudi, Marc Antonini, Pierre Kornprobst, Laurent U Perrinet
% \newblock A novel bio-inspired static image compression scheme for noisy data transmission over low-bandwidth channels.
% \newblock \emph{Acoustics Speech and Signal Processing (ICASSP)}, 2010.
%
% \item[A15]\years{2008} \fullcite{Davison08}
%
%
% \item[A14] \fullcite{Perrinet08spie}% _FA_
%
% \item[A13] \fullcite{Barthelemy08}
%
%
% \item[A12] \years{2007} \fullcite{Fischer07cv}
%
% \item[A11] \fullcite{Fischer07}
%
% \item[A10] \fullcite{Montagnini07}
%
% \item[A9] \fullcite{Perrinet07neurocomp}% _FA_
%
% \item[A8] \years{2004}~\fullcite{Perrinet04}% _FA_
%
% \item[A7] \fullcite{Perrinet04tauc}
% \item[A6] \fullcite{Perrinet03ieee}% _FA_
%
%
% \item[A5] \years{2003}~\fullcite{Perrinet03}% _FA_
%
%
% \item[A4] \years{2002}~\fullcite{Perrinet02sparse}% _FA_
%
% \item[A3] \fullcite{Perrinet01}% _FA_
%
% \item[A2] \fullcite{Delorme01}
%
% \item[A1] \fullcite{Perrinet02stdp} %  _FA_

% \end{enumerate}
%
%
% % \section{Chapitres d'ouvrage à comité de lecture}%
% \begin{enumerate}
%
\item[B6] \fullcite{Perrinet20}% _FA_
%
% \item[B5] \fullcite{Montagnini15bicv}
%
% \item[B4] \fullcite{Perrinet15bicv}% _FA_
%
% \item[B3] \fullcite{CristobalPerrinetKeil15bicv_chap1}
%
% \item[B2] \fullcite{Cessac07a}
%
% \item[B1] \fullcite{Perrinet07}% _FA_
%Laurent~U. Perrinet.
%\newblock Dynamical neural networks: modeling low-level vision at short
%  latencies.
%\newblock In \emph{Topics in Dynamical Neural Networks: From Large Scale Neural
%  Networks to Motor Control and Vision}, volume 142 of \emph{The European
%  Physical Journal (Special Topics)}, pages 163--225. Springer Verlag, Berlin /
%  Heidelberg, mar 2007.

%\item[1] { Laurent~U. Perrinet.}
%\newblock { Dynamical neural networks: modelling low-level vision at short latencies.}
%\newblock \emph{EPJ Special Topics "Topics in Dynamical Neural Networks: From Large Scale Neural Networks to Motor Control and Vision". Volume 142, pages 163--225.  mar 2007.}.
%%
 \end{enumerate}
%
% \section{Thèses et ouvrages}%

\begin{enumerate}

\item[-] \fullcite{CristobalPerrinetKeil15bicv}

 \end{enumerate}

%
% \item[-] \fullcite{Perrinet08neurocomp}
%
% \item[-] \fullcite{Cessac07}
%
% \item[-] \fullcite{Perrinet03these}
%  \end{enumerate}
%
%
% \section{Actes de conférences internationales à comité de lecture}%
% %~\citep{Perrinet08spie}
% \begin{etaremune}%enumerate}
%
% % # 2021
% \item \fullcite{Ladret21sfn}
% \item \fullcite{Ladret21crs}
% \item \fullcite{Jeremie21crs}
% \item \fullcite{Grimaldi21crs}
% \item \fullcite{Grimaldi21cosyne}
% \item \fullcite{Grimaldi21cbmi}
% \item \fullcite{Vergani21bernstein}
% % # 2020
% \item \fullcite{Ladret20aes}
% \item \fullcite{Franciosini20sigma}
% \item \fullcite{Franciosini20cosyne}
% \item \fullcite{Dauce20iwai}
% \item \fullcite{Boutin20sigma}
% % # 2019
% \item \fullcite{Ladret19sfn}
% \item \fullcite{Perrinet19nccd}
% \item \fullcite{BoutinFranciosiniChavaneRuffierPerrinet19sfn}
% \item \fullcite{FranciosiniPerrinet19cns}
% \item \fullcite{FranciosiniPerrinet19neurofrance}
% \item \fullcite{BoutinFranciosiniRuffierPerrinet19GdrRobotics}
%
% % # 2018
% \item \fullcite{BoutinFranciosiniRuffierPerrinet18DoctoralDay}
% \item \fullcite{BoutinFranciosiniRuffierPerrinet18itwist}
% \item \fullcite{DupeyrouxBoutinSerresPerrinetViollet18}
% \item \fullcite{FranciosiniPerrinet18cs}
% \item \fullcite{Ladret18gdr}
% \item \fullcite{Mansour18vss}
% \item \fullcite{Pasturel18}
% \item \fullcite{Pasturel18anemo}
% \item \fullcite{Pasturel18grenoble}
% \item \fullcite{Perrinet18gdr}
%
% % # 2017
% \item \fullcite{BoutinFranciosiniRuffierPerrinet17DoctoralDay}
% \item \fullcite{BoutinRuffierPerrinet17neurofrance}
% \item \fullcite{BoutinRuffierPerrinet17spars}
% \item \fullcite{Mansour17ecvp}
% \item \fullcite{Mansour17gdr}
% \item \fullcite{Pasturel17gdr}
% \item \fullcite{Perrinet17gdr}
% \item \fullcite{Damasse17vss}
%
% %\item \fullcite{}
% %\item \fullcite{}
% \item \fullcite{Perrinet16EUVIP}
% \item \fullcite{Mansour16sfn}
% \item \fullcite{Mansour16gdr}
% \item \fullcite{Damasse16ecvp}
% \item \fullcite{Mansour16ecvp}
% \item \fullcite{Damasse16vss}
% \item \fullcite{Montagnini16ecvp}
% \item \fullcite{Perrinet16networks}
% \item \fullcite{Taouali16areadne}
%
% \item \fullcite{Ravello15}
% \item \fullcite{Vacher15icms}
% \item \fullcite{Perrinet15eusipco}
% \item \fullcite{Montagnini15sfn}
% \item \fullcite{Taouali15vss}
% \item \fullcite{Damasse15vss}
% \item \fullcite{Danion15sfn}
% \item \fullcite{Taouali15icmns}
%
% \item \fullcite{Vacher14ihp}
% \item \fullcite{Rudiger14cosyne}%
% \item \fullcite{PerrinetBednar14vss}
% \item \fullcite{Simoncini14vss}
% \item \fullcite{KaplanKhoei14}
% \item \fullcite{Khoei14vss}
% \item \fullcite{Taouali14neurocomp}
% \item \fullcite{Taouali14areadne}
%
% \item \fullcite{Meso14vss}
% \item \fullcite{Khoei13cns}
% \item \fullcite{Perrinet13cns}
% \item \fullcite{Khoei13cns}%
% \item \fullcite{Meso13vss}%
% \item \fullcite{Perrinet13jffos}%
% \item \fullcite{Simoncini13vss}
%
% \item \fullcite{Perrinet12areadne}
% \item \fullcite{Masson12areadne}
% \item \fullcite{Khoei12sfn}
% \item \fullcite{Simoncini12coding}
% \item \fullcite{Simoncini12vss}
%
% \item \fullcite{Simoncini11Pattern}
% \item \fullcite{Simoncini11vss}
% \item \fullcite{Perrinet11sfn}
%
% \item \fullcite{Simoncini10vss}
% \item \fullcite{Perrinet10tauc}
% \item \fullcite{Perrinet10areadne}
% \item \fullcite{Bogadhi10vss}
% \item \fullcite{Simoncini10vss}
% \item \fullcite{Voges10neurocomp}
%
%
% \item \fullcite{Perrinet09vss}
% \item \fullcite{Perrinet09cosyne}
% \item \fullcite{Voges09cosyne}
% \item \fullcite{Voges09gns}
% \item \fullcite{Kremkow09gns}
% \item \fullcite{Yger09gns}
%
% \item \fullcite{Kremkow08sfn}
% \item \fullcite{Voges08neurocomp}
% \item \fullcite{Voges08}
% \item \fullcite{Kremkow08neurocomp}
% \item \fullcite{Perrinet08a}
% \item \fullcite{Perrinet08areadne}
% \item \fullcite{Perrinet08}
% \item \fullcite{Perrinet08spie}
%
% \item \fullcite{Davison07cns}
% \item \fullcite{Kremkow07cns}
% \item \fullcite{Montagnini07a}
% \item \fullcite{Montagnini07b}
% \item \fullcite{Perrinet07cns}
% \item \fullcite{Perrinet07mipm}
%
% \item \fullcite{Perrinet06ciotat}
% \item \fullcite{Perrinet06cns}
% \item \fullcite{Perrinet06neurocomp}
% \item \fullcite{Perrinet06fab}
% \item \fullcite{Perrinet06fens}
% \item \fullcite{Montagnini06neurocomp}
% \item \fullcite{Wohrer06}
%
% \item \fullcite{Perrinet05}
% \item \fullcite{Perrinet05a}
% \item \fullcite{Fischer05a}
% \item \fullcite{Fischer05}
% \item \fullcite{Redondo05}
%
% \item \fullcite{Fischer05a}
% \item \fullcite{Perrinet05}
%
% \item \fullcite{Perrinet05a}
%
% \item \fullcite{Perrinet02nsi}
% \item \fullcite{Perrinet02esann}
%
% \item \fullcite{Perrinet00}
%
% \end{etaremune}%
%
%
% \section{Conférences orales invitées}%
%
% \begin{etaremune}%enumerate}
%
% \item \fullcite{2022-03-22_SIAM-IS22}
% \item \fullcite{2022-01-12_NeuroCercle}
% \item \fullcite{2021-08-27_DDXL}
% \item \fullcite{2021-06-15_SMB}
% \item \fullcite{2021-05-20_NeuroFrance}
% \item \fullcite{2020-09-25_IRPHE}
% \item \fullcite{2020-09-14_IWAI}
% \item \fullcite{2020-04_UE-neurosciences-computationnelles}
% \item \fullcite{2020-01-20_atelier_sciences_cinema}
% \item \fullcite{2019-07-15_CNS}
% \item \fullcite{2019-05-23_Neurofrance}
% \item \fullcite{2019-04-18_JNLF}
% \item \fullcite{2019-04-05_BBCP_causal_kickoff}
% \item \fullcite{2019-03-25_HDR_RobinBaures}
% \item \fullcite{2019-01-18_LACONEU}
% \item \fullcite{2018-04-05_BCP_talk}
% \item \fullcite{2017-01-18_LACONEU}
% \item \fullcite{2017-06-28_Telluride}
% \item \fullcite{2016-07-07_EDP-proba}
% \item \fullcite{2016-10-13_LAW}
% \item \fullcite{2016-11-03_SIGMA}
% \item \fullcite{2016-10-26_FillatreBarlaudPerrinet16EUVIP}
% \item \fullcite{2016-10-26_Perrinet16EUVIP}
% \item \fullcite{2016-11-03_gdr}
% \item \fullcite{2015-11-05_Chile}
% \item \fullcite{2015-10-07_GDR-BioComp}
% \item \fullcite{2014-01-10_INTFest}
% \item \fullcite{2014-03-20_Manchester}
% \item \fullcite{2014-04-25_kaplan-beijing}
% \item \fullcite{2013-03-21_Marseille}
% \item \fullcite{2013-07-05_Cerco}
% \item \fullcite{2013-11-26_BrainScalesDemos}
% \item \fullcite{2012-01-12_VisionAtUcl}
% \item \fullcite{2012-01-24_Edinburgh}
% \item \fullcite{2012-01-27_FIL}
% \item \fullcite{2012-03-22_Juelich}
% \item \fullcite{2012-03-23_Juelich}
% \item \fullcite{2012-05-10_itwist}
% \item \fullcite{2011-07-02_NeuroMedTalk}
% \item \fullcite{2011-09-28_ermites}
% \item \fullcite{2011-10-05_BrainScalesESS}
% \item \fullcite{2011-11-15_sfn}
% \item \fullcite{2010-01-08_facets}
% \item \fullcite{2010-12-17_TaucTalk}
% \item \fullcite{2009-04-01_INT}
% \item \fullcite{2009-07-18_Kremkow09cnstalk}
% \item \fullcite{2009-11-30_vss}
% \item \fullcite{2008-02-01_toledo}
% \item \fullcite{2008-04-01_incm}
% \item \fullcite{2008-06-01_Ulm}
% \item \fullcite{2007-09-01_mipm}
% \item \fullcite{2007-12-01_rankprize}
% \item \fullcite{2006-01-01_neurocomp}
% \end{etaremune}%
%
%
% \section{Cours et actions de diffusion de la culture scientifique}%
%
% \begin{etaremune}%enumerate}
% \item \fullcite{Perrinet21hasard}
% \item \fullcite{Perrinet19temps}
% \item \fullcite{Perrinet19illusions}
% \item \fullcite{2019-01-10_PollyMaggoo}
% \item \fullcite{2019-01-14_LACONEU}
% \item \fullcite{2019-01-16_LACONEU}
% \item \fullcite{2019-01-17_LACONEU}
% \item \fullcite{2019-04-03_a_course_on_vision_and_modelization}
% \item \fullcite{2018-01-25_meetup-neuronautes}
% \item \fullcite{2018-02-01_BCP_INVIBE_fest}
% \item \fullcite{2018-03-26_cours-NeuroComp_FEP}
% \item \fullcite{2018-10-10_PollyMaggoo}
% \item \fullcite{2018-10-11_BioMorphisme}
% \item \fullcite{2017-01-19_LACONEU}
% \item \fullcite{2017-01-20_LACONEU}
% \item \fullcite{2017-06-30_Telluride}
% \item \fullcite{2017-11-15_ColloqueMaster}
% \item \fullcite{2017-11-17_FestivalInterferences}
% \item \fullcite{2017-11-24_NeurosciencesRobotique}
% \item \fullcite{2016-04-25_PollyMaggoo}
% \item \fullcite{2016-04-28_Mejanes}
% \item \fullcite{2016-11-20_PollyMaggoo}
% \item \fullcite{2010-04-14_OndesParalleles}
% \item \fullcite{Perrinet10DocSciences}
% \item \fullcite{2009-11-24_IntelligenceMecanique}
%
% \end{etaremune}%
%
% %%%%%%%%%%%%%%%%%%%%%%%%%%%%%%%%%%%%%%%%%%%%%%%%%%%%%%%%%%%
% \end{document} %
